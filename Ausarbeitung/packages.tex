\usepackage[T1]{fontenc} % T1-Fonts are better
\usepackage[utf8]{inputenc} % UTF8, bitte was sonst
%\usepackage{lmodern} % Eine etwas angenehmere Font
\usepackage{ucs} % Extended UTF-8 input support for LaTeX
\usepackage{cite} % Improved Citation-Handling
\usepackage{amsmath} % American Math Society Packete
\usepackage{amsfonts} % American Math Society Packete
\usepackage{amssymb} % American Math Society Packete
\usepackage{graphicx} % Erweiterter Support für Graphiken
\usepackage[ngerman]{babel} % Übersetze einige Literate auf Deutsch
\usepackage{enumerate} % Erweiterte Enumerate-Umgebungen
\usepackage{wrapfig} % Erlaubt es Text, Graphiken zu umfließen
\usepackage{caption} % % American Math Society Packete
\usepackage{listings}
\usepackage{subcaption} % Support for sub-captions (and sub-figures)
\usepackage{float} % Verbesserte floating-objects
%\usepackage{geometry} % Beinflusst das Seitenlayout..
%\geometry{a4paper, top=25mm, inner=25mm, outer=25mm, bottom=25mm, headsep=10mm, footskip=10mm}
\usepackage{framed} % Umrahmte Abschnitte
\usepackage[framed,hyperref,amsmath,thmmarks]{ntheorem}
\usepackage{tikz} % Für LaTeX Graphiken
\usetikzlibrary{decorations.markings,decorations.pathreplacing,shapes.geometric,automata,positioning,shadows} % Ein paar nützliche TikZ-Pakete
\usepackage{ stmaryrd } % St Mary Road symbols for theoretical computer science
\usepackage[hang]{footmisc}   % Für den Einzug bei Fußnoten
\setlength{\footnotemargin}{0pt} % Setze den Einzug für die Fußnoten

\usepackage{makeidx} % Index für Schlagwörter
\makeindex


%\usepackage[autostyle]{csquotes} % Erweitere Anführungszeichen

\makeindex


\usepackage[unicode,pdfmenubar,linktoc=all,hidelinks,bookmarks]{hyperref} % PDF-Verklinkungen ermöglichen

