

% Dictum schöner aussehen lassen
\renewcommand*{\dictumwidth}{.45\textwidth}

% Neue Absätze nicht einziehen, das zieht sonst grauenhaft aus.
\KOMAoption{parskip}{half-}

% Abkürzende Befehle:


\newcommand{\keyword}[1]{\textbf{#1}}
\newcommand{\probl}[1]{\textsc{#1}}



% Listings

\definecolor{gray}{rgb}{0.4,0.4,0.4}
\definecolor{darkred}{rgb}{0.6,0.0,0.0}
\definecolor{cyan}{rgb}{0.0,0.6,0.6}

\lstset{
  basicstyle=\ttfamily,
  columns=fullflexible,
  showstringspaces=false,
  commentstyle=\color{gray}\upshape
}
\lstdefinelanguage{XML}
{
  basicstyle=\ttfamily,
  morestring=[s]{"}{"},
  morecomment=[s]{?}{?},
  morecomment=[s]{!--}{--},
  commentstyle=\color{darkgreen},
  moredelim=[s][\color{black}]{>}{<},
  moredelim=[s][\color{red}]{\ }{=},
  stringstyle=\color{blue},
  identifierstyle=\color{darkred}
}

% Theoreme:
\makeatletter
\newtheoremstyle{breakNormalHeader}%
{\item[\rlap{\vbox{\hbox{\hskip\labelsep \theorem@headerfont
				##1\ ##2\theorem@separator}\hbox{\strut}}}]}%
{\item[\rlap{\vbox{\hbox{\hskip\labelsep \theorem@headerfont
				##1\ ##2\normalfont\unboldmath\ (##3)\theorem@separator}\hbox{\strut}}}]}
\newtheoremstyle{breakSCHeader}%
{\item[\rlap{\vbox{\hbox{\hskip\labelsep \theorem@headerfont
				##1\ ##2\theorem@separator}\hbox{\strut}}}]}%
{\item[\rlap{\vbox{\hbox{\hskip\labelsep \theorem@headerfont
				##1\ ##2\sc\unboldmath\ (##3)\theorem@separator}\hbox{\strut}}}]}
\makeatother

\theorembodyfont{\normalfont}
\theoremstyle{breakNormalHeader}

\newcounter{theoCounter} 
\numberwithin{theoCounter}{chapter}
\newcounter{exmpCounter}
\numberwithin{exmpCounter}{chapter}


%\newtheorem{definition}[theoCounter]{Definition}
%\newtheorem{lemma}[theoCounter]{Lemma}
%\newtheorem{theorem}[theoCounter]{Theorem}
%\newtheorem{korolar}[theoCounter]{Korollar}
%\newtheorem{satz}[theoCounter]{Satz}
%\newtheorem{fact}[theoCounter]{Fakt}
%\newtheorem{example}[exmpCounter]{Beispiel}

\theoreminframepreskip{0pt}
\theoreminframepostskip{0pt}
\theoremframepreskip{0cm}
\theoremframepostskip{0cm}
\tikzstyle{thmbox} = [rounded corners, fill=gray!10, inner sep=3pt]
\newcommand\thmbox[1]{%
	\noindent\begin{tikzpicture}%
	\node [thmbox] (box){%
		\begin{minipage}{1.00\textwidth}%
		#1%
		\end{minipage}%
	};%
	\end{tikzpicture}}
\let\theoremframecommand\thmbox
\newshadedtheorem{definition}[theoCounter]{Definition}
\tikzstyle{thmboxx} = [fill=gray!10, inner sep=3pt]
\newcommand\thmboxx[1]{%
	\noindent\begin{tikzpicture}%
	\node [thmboxx] (box){%
		\begin{minipage}{1.00\textwidth}%
		#1%
		\end{minipage}%
	};%
	\end{tikzpicture}}
\let\theoremframecommand\thmboxx
\newshadedtheorem{lemma}[theoCounter]{Lemma}
\newshadedtheorem{theorem}[theoCounter]{Theorem}
\newshadedtheorem{korolar}[theoCounter]{Korollar}
\newshadedtheorem{satz}[theoCounter]{Satz}
\newshadedtheorem{fact}[theoCounter]{Fakt}
\newtheorem{example}[exmpCounter]{Beispiel}

\theoremstyle{breakSCHeader}
\theoremheaderfont{\bfseries\scshape}
\newframedtheorem{problem}[theoCounter]{Problem}
%\newtheorem{problem}[theoCounter]{Problem}

\theoremstyle{breakNormalHeader}
\theoremheaderfont{\normalfont}\theorembodyfont{\upshape}
\theoremstyle{nonumberplain}
\theoremseparator{.}
\theoremsymbol{\rule{1ex}{1ex}}
\newtheorem{proof}{Beweis}

% Nummerierte Formeln nach Kapiteln
\renewcommand{\theequation}{\arabic{chapter}.\arabic{equation}}

%\baselineskip14pt\normalfont
%\renewcommand{\baselinestretch}{1}
\setlength\abovedisplayshortskip{0pt}
\setlength\belowdisplayshortskip{0pt}
\setlength\abovedisplayskip{2pt}
\setlength\belowdisplayskip{2pt}